% This is the ADASS_template.tex LaTeX file, 26th August 2016.
% It is based on the ASP general author template file, but modified to reflect the specific
% requirements of the ADASS proceedings.
% Copyright 2014, Astronomical Society of the Pacific Conference Series
% Revision:  14 August 2014

% To compile, at the command line positioned at this folder, type:
% latex ADASS_template
% latex ADASS_template
% dvipdfm ADASS_template
% This will create a file called aspauthor.pdf.}

\documentclass[11pt,twoside]{article}


\newcommand{\yt}{\texttt{yt}}

% Do NOT use ANY packages other than asp2014. 
\usepackage{asp2014}

\aspSuppressVolSlug
\resetcounters

% References must all use BibTeX entries in a .bibfile.
% References must be cited in the text using \citet{} or \citep{}.
% Do not use \cite{}.
% See ManuscriptInstructions.pdf for more details
\bibliographystyle{asp2014}

% The ``markboth'' line sets up the running heads for the paper.
% 1 author: "Surname"
% 2 authors: "Surname1 and Surname2"
% 3 authors: "Surname1, Surname2, and Surname3"
% >3 authors: "Surname1 et al."
% Replace ``Short Title'' with the actual paper title, shortened if necessary.
% Use mixed case type for the shortened title
% Ensure shortened title does not cause an overfull hbox LaTeX error
% See ASPmanual2010.pdf 2.1.4  and ManuscriptInstructions.pdf for more details
\markboth{Goldbaum}{Extracting Insights from Astrophysics Simulations}

\begin{document}

\title{Extracting Insights from Astrophysics Simulations}

% Note the position of the comma between the author name and the 
% affiliation number.
% Author names should be separated by commas.
% The final author should be preceded by "and".
% Affiliations should not be repeated across multiple \affil commands. If several
% authors share an affiliation this should be in a single \affil which can then
% be referenced for several author names.
% See ManuscriptInstructions.pdf and ASPmanual2010.pdf 3.1.4 for more details
\author{Nathan~J.~Goldbaum$^1$}
\affil{$^1$National Center for Supercomputing Applications, University of
  Illinois at Urbana-Champaign, Urbana, IL, USA; \email{ngoldbau@illinois.edu}}

% This section is for ADS Processing.  There must be one line per author.
\paperauthor{Nathan~J.~Goldbaum}{ngoldbau@illinois.edu}{0000-0001-5557-267X}{University
of Illinois at Urbana-Champaign}{National Center for Supercomputing Applications}{Urbana}{IL}{61801}{USA}


\begin{abstract}
Simulations inform all aspects of modern astrophysical research, ranging in scale from 1D and 2D test problems that can run in seconds on an astronomer's laptop all the way to large-scale 3D calculations that run on the largest supercomputers, with a spectrum of data sizes and shapes filling the landscape between these two extremes. I review the diversity of astrophysics simulation data formats commonly in use by researchers, providing an overview of the most common simulation techniques, including pure N-body dynamics, smoothed particle hydrodynamics (SPH), adaptive mesh refinement (AMR), spectral methods, and unstructured meshes. Additionally, I highlight methods for incorporating physical phenomena that are important for astrophysics, including chemistry, magnetic fields, radiative transport, and "subgrid" recipes for important physics that cannot be directly resolved in a simulation. In addition to the numerical techniques, I also discuss the communities that have developed around these simulation codes and argue that increasing use and availability of open community codes has dramatically lowered the barrier to entry for novice simulators. Extracting scientific results from astrophysical simulation data requires detailed knowledge of the underlying data structures and data formats, as well as the semantic meaning of the data in relation to the physics problem posed by the simulation. As a solution to this problem, I present \yt\, a community-developed python library for analyzing and visualizing simulation data.
\end{abstract}

\section{Introduction}
Blah blah.

\section{Another Section}
References must be provided in BibTeX format, in a .bib file, and should usually be referenced using \verb"\citet" or \verb"\citep". The file example.bib supplied with this template is taken from an ADASS 2015 paper, and includes references to previous ADASS proceedings 
\citep[such as][]{1999ASPC..172..487P} and to papers in what was then the current 2015 proceedings (e.g.\ \citet{O11-4_adassxxv}). Note that the `TBD' entries that appear for papers in the current proceedings will be dealt with by the ADASS editors when the final volume is produced. The example .bib file has a large number of references unused by this template file; such unused references have been left in as an example, but should be removed before a paper is submitted.

\acknowledgements The ASP would like to thank the dedicated researchers who are publishing with the ASP.  It will make things a lot easier if you keep this text on the same line as the \verb"\acknowledgements" command.


\bibliography{example}  % For BibTex

\end{document}
