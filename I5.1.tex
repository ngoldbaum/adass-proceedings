% This is the ADASS_template.tex LaTeX file, 26th August 2016.
% It is based on the ASP general author template file, but modified to reflect the specific
% requirements of the ADASS proceedings.
% Copyright 2014, Astronomical Society of the Pacific Conference Series
% Revision:  14 August 2014

% To compile, at the command line positioned at this folder, type:
% latex ADASS_template
% latex ADASS_template
% dvipdfm ADASS_template
% This will create a file called aspauthor.pdf.}

\documentclass[11pt,twoside]{article}


\newcommand{\yt}{\texttt{yt}}

% Do NOT use ANY packages other than asp2014. 
\usepackage{asp2014}

\aspSuppressVolSlug
\resetcounters

% References must all use BibTeX entries in a .bibfile.
% References must be cited in the text using \citet{} or \citep{}.
% Do not use \cite{}.
% See ManuscriptInstructions.pdf for more details
\bibliographystyle{asp2014}

% The ``markboth'' line sets up the running heads for the paper.
% 1 author: "Surname"
% 2 authors: "Surname1 and Surname2"
% 3 authors: "Surname1, Surname2, and Surname3"
% >3 authors: "Surname1 et al."
% Replace ``Short Title'' with the actual paper title, shortened if necessary.
% Use mixed case type for the shortened title
% Ensure shortened title does not cause an overfull hbox LaTeX error
% See ASPmanual2010.pdf 2.1.4  and ManuscriptInstructions.pdf for more details
\markboth{Goldbaum}{Extracting Insights from Astrophysics Simulations}

\begin{document}

\title{Extracting Insights from Astrophysics Simulations}

% Note the position of the comma between the author name and the 
% affiliation number.
% Author names should be separated by commas.
% The final author should be preceded by "and".
% Affiliations should not be repeated across multiple \affil commands. If several
% authors share an affiliation this should be in a single \affil which can then
% be referenced for several author names.
% See ManuscriptInstructions.pdf and ASPmanual2010.pdf 3.1.4 for more details
\author{Nathan~J.~Goldbaum$^1$}
\affil{$^1$National Center for Supercomputing Applications, University of
  Illinois at Urbana-Champaign, Urbana, IL, USA; \email{ngoldbau@illinois.edu}}

% This section is for ADS Processing.  There must be one line per author.
\paperauthor{Nathan~J.~Goldbaum}{ngoldbau@illinois.edu}{0000-0001-5557-267X}{University
of Illinois at Urbana-Champaign}{National Center for Supercomputing Applications}{Urbana}{IL}{61801}{USA}


\begin{abstract}
Simulations inform all aspects of modern astrophysical research, ranging in scale from 1D and 2D test problems that can run in seconds on an astronomer's laptop all the way to large-scale 3D calculations that run on the largest supercomputers, with a spectrum of data sizes and shapes filling the landscape between these two extremes. I review the diversity of astrophysics simulation data formats commonly in use by researchers, providing an overview of the most common simulation techniques, including pure N-body dynamics, smoothed particle hydrodynamics (SPH), adaptive mesh refinement (AMR), and unstructured meshes. Additionally, I highlight methods for incorporating physical phenomena that are important for astrophysics, including chemistry, magnetic fields, radiative transport, and "subgrid" recipes for important physics that cannot be directly resolved in a simulation. In addition to the numerical techniques, I also discuss the communities that have developed around these simulation codes and argue that increasing use and availability of open community codes has dramatically lowered the barrier to entry for novice simulators. Extracting scientific results from astrophysical simulation data requires detailed knowledge of the underlying data structures and data formats, as well as the semantic meaning of the data in relation to the physics problem posed by the simulation. As a solution to this problem, I present \yt, a community-developed Python library for analyzing and visualizing simulation data.
\end{abstract}

\section{Introduction}
Direct 3D numerical simulations of astrophysical phenomena are a key driver of theoretical prediction in 21st century astrophysics. By offering a dynamical view of phenomena that are unobservable on human timescales, numerical simulations allow researchers to extract unique insights that are unavailable from other approaches due to the complexity of the underlying physical systems. Due to the nature of the numerical discretizations driving the simulation codes, increasing computer power will naturally bring more and more phenomena within the envelope of computation feasability. Coupled with improved computational methods and modeling techniques this means nearly all fields in astrophysics are informed by direct numerical simulations at some level, and in many fields at a very high level.

It is therefore not surprising that the astrophysics simulation software ecosystem is quite diverse. Different systems lend themselves to different simulation techniques, results are validated by being reproduced from completely independent codebases, and individual research groups are incentivized to develop software both to produce new results and in some cases to make proprietary use of new technology to produce novel research results. Increasingly, open research communities are developing, producing simulation software that can be used by anyone with access to the necessary computational resources to run a simulation.

In this paper I will review the ecosystem of astrophysical simulation software and simulation techniques. This will by no means be a comprehensive review. Instead, I will primarily focus on generally available community research software. I will also mention several propietary research codes that have made a substantial impact on the research landscape in terms of published results. I will also attempt to focus on codes that are still used to generate research results, leaving older software that is no longer commonly used to historical reviews. The review will be organized according to the simulation technique used by the software. First, Section~\ref{nbody} in I will discuss pure N-body and smoothed particle hydrodynamics (SPH) codes. Next, in Section~\ref{amr} I will discuss grid-based hydrodynamics codes, with a focus on adaptive mesh refinement codes. In Section~\ref{usm} I will discuss moving mesh hydrodynamics codes. Finally, in Section~\ref{yt} I will introduce \yt, a community-developed Python library for analyzing and visualizing the outputs of these simulation codes.

\section{N-body Methods}
\label{nbody}

In this approach, an astrophysical system is discretized into a system of self-gravitating particles which interact via their mutual gravitational attraction. In most cases the particles are simply test particles which are used to sample the gravitational force of the system, but each individual particle does not necessarily have any physical meaning. N-body methods are Lagrangian and thus well-suited for astrophysical problems that tend to have large dynamic ranges in the scale of structures. Rather than wasting computation on regions in which not much is happening, work naturally flows to where the mass is contained. N-body methods in astrophysics can be further subdivided in pure N-body methods and SPH methods, which can include the effect of gas dynamics. We describe these two approaches below.

Using pure N-body techniques to understand physical systems has a rich history \citep[see e.g.\ ][]{holmberg1941, hoerner1960, peebles1970, press1974}, in particular for astrophysical systems where the influence of gas dynamics is negligible. In some cases, such as simulations of star clusters \citep{wang2016}, or even galaxies \citep{bedorf2014}, the masses of particles in the simulation approach the masses of the physical bodies that make up the real gravitational system.

Since gravity is a long-range force, this means that the gravitational force calculation for any given particle depends on the masses and positions of every other particle in the simulation. Therefore, the worst-case naive algorithm that evaluates the pairwise gravitational force for all particles is $O(N^2)$. State-of-the-art N-body calculations include as many as a trillion particles and are run on clusters containing thousands of compute nodes --- the naive pairwise algorithm would simply not be feasible to run on simulations of this size. Instead, N-body codes make use of some form of spatial acceleration to conver the calculation of the gravitational force into an $O(N \log N)$ calculation. Often, this involves depositing the N-body data into a mesh and then making use of a fast Fourier transform (FFT) to solve the Poisson equation on the mesh, this is known as a Particle-mesh (PM) algorithm \citep{hockney1988}. The accelerations at the particle locations can then be calculated by interpolating from the accelerations at the mesh locations. Another alternative is to calculate the gravitational acceleration by making use of an octree data structure. Particles contained in octree zones that are sufficiently far away (usually quantified in terms of an opening angle relative to the particle under consideration) are grouped together, and the collective effect of distant particles is averaged, possibly using a multipole expansion. This algorithm is originally due to \citet{barnes1986}, and octrees used for a gravitational force calculation are known as a Barnes-Hut tree. These two approaches can be combined to produce a TreePM algorithm \citep{bagla2002}, with a tree used to calculate the influence of nearby particles, and mesh deposition used for the most distant particles. This allows much easier implementation of periodic boundary conditions useful for cosmology simulations while retaining the high force resolution of the Barnes-Hut algorithm.

Including the effects of gas dynamics is critical for understanding many processes in astrophysics. Many problems include the coupled effects of particle dynamics and gas dynamics, so it becomes very appealing to directly include gas dynamics in an N-body model. \citet{lucy1977} and \citet{gingold1977} introduced the concept of smoothed particle hydrodynamics, where N-body particles are used as Lagrangian markers to sample an underlying continuous field. Rather than defining the properties of the gas based on local properties of any given particle, the properties of the gas are calculated based on a weighted average of the value of a field at the locations of nearby particles. The weighting function, known as the smoothing kernel, defines a finite compact region where particles may interact with each other. After calculating the gravitational forces (if any) using the N-body algorithms described above, the hydrodynamic forces may be calculated by solving a discretized version of the equiations of fluid dynamics, where the individual terms in the equations are calcualted by averaging over a smoothing kernel. See \citet{price2012} for a detailed recent review of modern SPH formalism.

In recent years, SPH algorithms have come under criticism for not being able to resolve certain fluid instabilities \citep{agertz2007} and artificially suppressing mixing \citep{read2010}. This has led to the development of alternative SPH formalisms \citep{abel2011, hopkins2013}, as well as moving mesh methods (see Section~\ref{usm} for more discussion).

The Gadget family of SPH codes is quite possibly the most commonly used astrophysical simulation code. Originally created in 1998 by Volker Springel as part of his PhD project, Gadget-1 was released publicly in 2000 \citep{springel2001}. Later, the next-generation Gadget-2 code was released, including a complete rewrite in C, and improved algorithms, including an entropy-conserving approach for forumulating SPH, and a TreePM implementation \citep{springel2005b} that makes Gadget-2 suitable for large-scale N-body calculations of structure formation. This capability was used to run the Millenium simulations \citep{springel2005a}, at the time the largest pure N-body simulation of structure formation in the universe, using more than $10^{10}$ particles. Finally, Gadget-3 was developed afterwards but was never released publicly. It includes an improved domain decomposition algorithm . In addition, Gadget has been used extensively to study astrophysical phenomena at many scales, including the impact of gas on cosmological structure formation \citep{keres2009, schaye2015}, isolated galaxies \citep{springel2005c} and galaxy collisions \citep{robertson2006}, active galactic nucleus (AGN) feedback \citep{sijacki2007}, the formation of the first stars \citep{clark2011}, and simulations of the formation of the Moon \citep{jackson2012}.

Gadget-2 was released under the terms of the GNU General Public License v2 in 2005. It is available as a tarball at \url{http://wwwmpa.mpa-garching.mpg.de/gadget/}. The public release includes an adiabatic hydrodynamics module, self gravity based on a Barnes-Hut tree or TreePM, and can be used in comoving coordinates, making it suitable for cosmological simulations. The public release does not include any additional physics modules. While there were physics modules developed for the original version of the code, they were kept private. This lack of physics modules in the public version has led to the development of many independent private versions of Gadget-2 that include alternate implementations of physics such as cooling and heating processes, star formation and feedback, AGN feedback, and sink particles. Rather than any single Gadget community, there are instead balkanized islands of code development within individual research groups.

Tipsy (\url{https://github.com/N-BodyShop/tipsy}) is a paired data format and visualization tool originally written by Tom Quinn in the late 80's. The data format is shared by a number of simulation codes whose lineage traces back to the N-body shop research group at the University of Washington.

PKDGRAV \citep{stadel2001} is a pure N-body dynamics code. Rather than making use of an octree to calculate the gravitational force, it instead makes use of a parallel $k$-d tree. Compared with an octre, a $k$-d tree has better data balance and can be constructed more efficiently. This property enabled extremely high resolution simulations of galaxy formation in $\Lambda$-CDM cosmology, such as the {\it via lactea\/} simulations \citep{diemand2007, diemand2008}. More recently, PKDGRAV was ported to GPUs \citep{potter2017} and made publicly available (\url{https://bitbucket.org/dpotter/pkdgrav3/}), although without a license declaration. In addition, PKDGRAV is the basis of the other simulation codes in the Tipsy family, which include the effects of hydrodynamics via SPH particles.

Gasoline \citep{wadsley2004} is a privately developed SPH code originally built on top of PKDGRAV.\@ Similar to how PKDGRAV was used most impressively to elucidate the details of Milky Way mass dark matter halos, gasoline has been used quite heavily to understand the details of disk galaxy formation using high resolution simulations. An early example was published by \citet{governato2004}, although these models suffered spurious angular momentum loss due to poor resolution in the disk and inefficient feedback.  Later gasoline simulations employed higher resolution and improevd physical models, including the blastwave feedback model of \citet{stinson2006} and metal line cooling and metal mixing models of \citet{shen2010}, produced more realistic galaxy models, including the simulations described by \citet{guedes2011}, \citet{pontzen2012}, and \citet{zolotov2012}. More recently, the next generation version: Gasoline2 was described in \citet{wadsley2017}. This version includes an updated SPH formalism which eliminates many of the drawbacks of ``traditional'' SPH described above.

Recently, the physical models available in the Gasoline codebase were incorporated into the ChaNGA code \citep{jetley2008}, a research codebase built on top of the \texttt{Charm++} message passing library. Charm++ enables ChaNGa to scale to higher CPU counts and obtain improved efficiency by enabling asynchronous calculation of the various steps in the simulation timestep by the compute nodes. ChaNGa is publicaly available at \url{https://github.com/N-BodyShop/changa}. The public release of ChaNGa on GitHub does not include a license declaration and Charm++ has a license that forbids commercial use.

Other public SPH clodes include SEREN \citep[\url{http://dhubber.github.io/seren},][]{hubber2011} and PHANTOM \citep[\url{https://phantomsph.bitbucket.io/},][]{price2017}. SEREN was released under the GPLv2. It has been heavily used in simulations of the interstellar medium and star cluster formation and includes prescriptions for sink particles \citep{walch2013} as well as timestepping schemes suitable for tracking N-body interactions in clusters. PHANTOM is available under the GPLv3 and includes a number of advanced SPH features, and subgrid models useful for studying the ISM and galaxies.

\section{Adaptive Mesh Refinement}
\label{amr}

\subsection{Enzo}
\label{enzo}

\subsection{RAMSES}
\label{ramses}

\subsection{Boxlib and Chombo families}
\label{boxlib}

\subsection{FLASH}
\label{flash}

\subsection{Cactus}
\label{cactus}

\subsection{Athena and Athena++}

\subsection{GAMER}

\section{Moving Mesh}
\label{usm}

\subsection{AREPO}
\label{arepo}


\subsection{GIZMO}
\label{gizmo}

\section{Analysis and Visualization with \yt\ }
\label{yt}

\bibliography{I5.1.bib}

\end{document}
